\documentclass[runningheads,a4paper]{report}
\usepackage{listings}
\usepackage{color}
\usepackage{xcolor}

\lstnewenvironment{code}
    {\lstset{%
        basicstyle=\ttfamily,
        escapechar=~,
        morekeywords={foo,bar},
        keywordstyle=\color{green},
        literate={>}{{\textcolor{blue}{>}}}1}
    }
    {}

\lstnewenvironment{codeoutput}
    {\lstset{basicstyle=\ttfamily\color{black}}}
    {}


\lstset{breaklines=true}


% Title Page
\title{CG2271 RTOS Lab 7 --- Processes in Unix}
\author{Divyanshu Arora(U096857U), Omer Iqbal(A0074933Y)}


\begin{document}
\maketitle
\lstset{language=C,
  label=Output, numbers=left, frame=shadowbox,
  rulesepcolor=\color{black}, numberstyle=\color{gray}}
\section*{Q1}

\subsection*{PID}

A \texttt{PID}, or a \textit{process identifier}, is a number used by
OS kernels to uniquely identify a process.

Since they are unique, they are used in calls to functions (syscalls,
or any other commands that wrap around them, \texttt{renice} for
instance) to identify \& operate on processes.

\subsection*{Identifying Parent process's parent}

We can use the \texttt{getppid()} syscall to obtain the
\texttt{pid\_t} (a typedef to \textit{int} on most system)
representation of the pid of the
invoking process's parent.

Therefore, we modify our program to be---
\lstset{caption=Adding a getppid() call, language=inform}
\begin{lstlisting}
  //[...]
  if ((pid=fork()))
    {
      printf("Process ID of parent's parent: %d\n", getppid());
      printf("Process ID of child: %d\n", pid);
      //[...]
\end{lstlisting}


On a sample run, we obtain the output---
\lstset{caption=Sample output, rulecolor=\color{black}}
\begin{codeoutput}
Process ID of parent's parent: 6246
Process ID of child: 15252
Parent: i=0 k=15
[truncated for brevity]
\end{codeoutput}

Some noob shell-fu later, we find the parent's name, its ppid, etc. 
\lstset{caption=Exhibit `A'}
\begin{code}
--- lab7 (master) > ps aux | grep 6246
div    6246  0.7  0.2 387020 41236 pts/3    Sl+  02:20   0:08 emacs
\end{code}

\section*{Q2}

The values of \textit{k} printed by the parent and the child are independent.

This is because the \texttt{fork()} call creates a copy of the parent's context for the child (well, technically, the copy is made later owing to copy-on-write optimizations). As such, when the parent increments \textit{k}, it doesn't affect the copy of \textit{k} held by the child.

\section*{Q3}

The output \textit{can} become mixed because when the parent (or the child) sleeps, the OS scheduler is not restricted from scheduling the other process for execution.

The child process continues to execute after the parent process completes. This is because on \textit{GNU/Linux},the system I'm using, child processes aren't killed when their parent exits. Instead, typically, the \texttt{init} process (pid 1) will take over parental duties for the process, and will reap its exit code.

\section*{Q4}

Added a \texttt{wait()} call to the \textit{parent}'s code.

\lstset{caption=git-diff of changes}
\lstinputlisting{q4.diff}

\section*{Q5}

The program forks off a child, which prints a message containing 2 variables populated by reading a pipe from the parent process.

\section*{Q6}

The \texttt{sizeof()} function returns a value of type \texttt{size\_t} that represents the number of bytes used to store the variable (or type) that was supplied as an argument.

\section*{Q7}

\lstset{caption=Using parent and a child to count the primes among randomly generated numbers, language=C}
\lstinputlisting{lab7c.c}


\end{document}
